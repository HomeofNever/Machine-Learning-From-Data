\documentclass{article}
\usepackage[utf8]{inputenc}
\usepackage{datetime}
\usepackage{enumerate}
\usepackage{textcomp}
\usepackage{amsmath}
\usepackage{tikz}
\usetikzlibrary{arrows}
\usepackage{graphicx}
\usepackage{amssymb}
\graphicspath{ {./images/} }
   
\title{\bf \Large ASSIGNMENT 4}
\author{Xinhao Luo}
\date{\today}

\def\math#1{$#1$}

\setlength{\textheight}{8.5in}
\setlength{\textwidth}{6.5in}
\setlength{\oddsidemargin}{0in}
\setlength{\evensidemargin}{0in}
\voffset0.0in

\begin{document}
\maketitle
\medskip

\section{Exercise 2.4}

\begin{enumerate}[a)]
    \item Let's set the input matrix X as 
        \begin{equation}
            X = \begin{bmatrix}
                    1 & 0 & 0 & \dots & 0 \\
                    1 & 1 & 0 & \dots & 0 \\
                    1 & 0 & 1 & \dots & 0 \\
                    \vdots & \vdots & \vdots & \vdots & \vdots\\
                    1 & 0 & 0 & \dots & 1
                \end{bmatrix}
        \end{equation}  
    where X is a non-singular matrix (square matrix) of \math{(d + 1) * (d + 1)}, so \math{X} is invertible.  \\
     As X is a invertible, so we may compute \math{X^{-1}} via \math{w = \math{X}^{-1}y}, and we have \math{Xw = XX^{-1}y = y}, where \math{y} is any dichotomy. For any dichotomy there is a solution \math{w}, thus any dichotomy on these data points can be shattered, \math{d_{vc} \geq d + 1}.
    \item Since any new vector will be linearly dependent of the \math{d + 1} vector, there will always be \math{x_{d + 2} = c_0x_0 + c_1x_1 + ... + c_{d + 1}x_{d + 1}}. If we have a dichotomy that \math{w^Tx_nc_n < 0}, then \math{sign(w^Tx_{d + 2})} will always be -1, and part of the dichotomy (+1) cannot be implemented. Thus for \math{N \geq d + 2}, \math{m_H(N) < 2^N}, which leads to \math{d_{vc} \leq d + 1}
\end{enumerate}

\section{Problem 2.3}

\begin{enumerate}[a)]
    \item For N points, the line has split by \math{N + 1} region, for each of the positive and negative e rays, so the max number of dichotomies is \math{2 * (N + 1) - 2 = 2N} (-2 exclude all positive/negative points situation being counted twice on the \math{-\infty, \infty} region). With \math{m_H(N) = 2N}, we have:
        \begin{itemize}
            \item \math{m_H(2) = 4 = 2^2}
            \item \math{m_H(3) = 6 < 2^3}
        \end{itemize}
    Thus, the \math{d_{vc} = 2}
    \item There are two cases:
        \begin{enumerate}[1)]
            \item If the interval covers parts of the rightmost the leftmost end, then the count is the same as part a), \math{2N}
            \item If the interval is completely within the \math{N - 1} range, then the number of dichotomies is \math{2 * { N - 1 \choose 2 } = (N - 1) * (N - 2)}
        \end{enumerate}
        Thus, we have \math{m_H(N) = (N - 1)(N - 2) + 2N = N^2 - N + 2}, and we will have
        \begin{itemize}
            \item \math{m_H(3) = 8 = 2^3}
            \item \math{m_H(4) = 14 < 2^4}
        \end{itemize}
        Thus, \math{d_vc = 3}
    \item This case has the same pattern of the positive interval (Example 2.2 Case 2), where each point can be assigned to a real number and t he concentric sphere is corresponded to the positive interval over the real number values. \math{d_{vc} = 2}
\end{enumerate}

\section{Problem 2.8}

A growth function with break point can be bonded by a polynomial function of N or if it does not have a break point, it would be \math{2^N}. Thus, \math{1 + N}, \math{1 + N + \frac{N(N - 1)}{2}}, \math{2^N} are possible grow functions. 

The rest does not satisfied the bound: 

\begin{itemize}
    \item [\math{2^{\sqrt{N}}}] We will have \math{m_H(1) = 2 = 2^1} and \math{m_H(2) = 2 < 2^2}, but \math{m(20) \nless 20^{2 - 1} + 1}
    \item [\math{2^{N / 2}}] We will have \math{m_H(0) = 1 = 2^1} and \math{m_H(1) < 2^1}, but \math{m(3) \nless 3^{1 - 1} + 1}
    \item [(3)] For \math{1 + N + \frac{N(N - 1)(N - 2)}{6}}, We will have \math{m_H(1) = 2 = 2^1} and \math{m_H(2) = 3 < 2^2}, but \math{m(3) \nless 2^{2 - 1} + 1}
\end{itemize}

\section{Problem 2.10}

We may consider it as two groups of N points, so that for each group, the max number of dichotomies is \math{m_H(N)}. When it comes together of 2N points, \math{m_H(2N)} is at most \math{m_H(N) * m_H(N)}. Thus we have \math{m_H(2N) \leq m_H(N)^2}

The bound will then be: 

\begin{equation}
    E_{out}(g) \leq E_{in} + \sqrt{\frac{8}{N}ln(\frac{4m_H(N)^2}{\rho})}
\end{equation}

\section{Problem 2.12}

From 2.13, we know that 

\begin{equation}
    N \geq \frac{8}{\epsilon^2}ln(\frac{4((2N)^{d_{vc}} + 1)}{\delta})
\end{equation}

Here we have \math{d_{vc} = 10}, \math{\epsilon = 0.05}, \math{\delta = 1 - 0.95 = 0.05}, and set \math{N = 10000}. Thus: 

\begin{equation}
    \begin{split}
        N & \geq \frac{8}{0.05^2}ln(\frac{4((2 * 10000)^{10} + 1)}{0.05}) = 330935 \\
        N & \geq \frac{8}{0.05^2}ln(\frac{4((2 * 330935)^{10} + 1)}{0.05}) = 442913 \\
        N & \geq \frac{8}{0.05^2}ln(\frac{4((2 * 442913)^{10} + 1)}{0.05}) = 452240 \\
        N & \geq \frac{8}{0.05^2}ln(\frac{4((2 * 452240)^{10} + 1)}{0.05}) = 452907 \\
        N & \geq \frac{8}{0.05^2}ln(\frac{4((2 * 452907)^{10} + 1)}{0.05}) = 452954 \\
        N & \geq \frac{8}{0.05^2}ln(\frac{4((2 * 452954)^{10} + 1)}{0.05}) = 452957 \\
        N & \geq \frac{8}{0.05^2}ln(\frac{4((2 * 452957)^{10} + 1)}{0.05}) = 452957
    \end{split}
\end{equation}

So we need \math{N \geq 452957}

\end{document}
